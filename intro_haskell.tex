\documentclass{beamer}

%\usepackage[utf8]{inputenc}
%\usepackage[T1]{fontenc}
%\usepackage{lmodern}
%\usepackage{graphicx}
\usepackage[english]{babel}
%\usepackage{array}
%\usepackage{multirow}
%\usepackage{caption}
%\usepackage{fixltx2e}
\usepackage{listings}
%\setbeamertemplate{bibliography item}{[\theenumiv]}
\usepackage[outputdir=figs]{diagrams-latex}
\setbeamertemplate{note page}[plain]
\setbeameroption{show notes}

\usetheme[]{Berkeley} %backgroundimagefile=figs/logo.png, useblacktitletext

\titlegraphic{
	\includegraphics[width=2cm]{figs/logo}
   \hspace*{4.75cm}
   \includegraphics[width=2cm]{figs/logo_wlhn}
}

\newcommand {\framedgraphic}[2] {
    \begin{frame}{#1}
        \begin{center}
            \includegraphics[width=\textwidth,height=0.8\textheight,keepaspectratio]{#2}
        \end{center}
    \end{frame}
}

\begin{document}

\lstset{
  language=Haskell,
  keywordstyle=\color{blue},
  basicstyle=\small\sffamily,
  columns=fullflexible,
  showstringspaces=false,
  mathescape=true,
  captionpos=b
}

\title{Introduction to Haskell}
\subtitle{West London Hack Night}
\author{Dominic Steinitz}

\date{23 April 2014}

\maketitle

\begin{frame}
  \frametitle{Table of Contents}
  \begin{columns}[]
   \begin{column}[]{5cm}
    \tableofcontents[]
   \end{column}
   \begin{column}[]{5cm}
    \includegraphics[width=1\linewidth]{figs/xkcd}
   \end{column}
  \end{columns}
\end{frame}

\section{Haskell in a Nutshell}

\begin{frame}{Hindley Milner Principal Type Inference Algorithm}

$$
\bar{\Gamma}(\tau) = \forall\ \hat{\alpha}\ .\ \tau \quad\quad \hat{\alpha} = \textrm{free}(\tau) - \textrm{free}(\Gamma)
$$

$$
\begin{array}{cl}
\displaystyle\frac{x:\sigma \in \Gamma \quad \sigma \sqsubseteq \tau}{\Gamma \vdash x:\tau}&[\mathtt{Var}]\\ \\
\displaystyle\frac{\Gamma \vdash e_0:\tau \rightarrow \tau' \quad\quad \Gamma \vdash e_1 : \tau }{\Gamma \vdash e_0\ e_1 : \tau'}&[\mathtt{App}]\\ \\
\displaystyle\frac{\Gamma,\;x:\tau\vdash e:\tau'}{\Gamma \vdash \lambda\ x\ .\ e : \tau \rightarrow \tau'}&[\mathtt{Abs}]\\ \\
\displaystyle\frac{\Gamma \vdash e_0:\tau \quad\quad \Gamma,\,x:\bar{\Gamma}(\tau) \vdash e_1:\tau'}{\Gamma \vdash \mathtt{let}\ x = e_0\ \mathtt{in}\ e_1 :  \tau'}&[\mathtt{Let}]
\end{array}
$$

\end{frame}


\section{Motivation}

\begin{frame}[fragile]
\frametitle{What does this code do?}

\begin{lstlisting}[language=C, basicstyle=\tiny]
void f(int a[], int lo, int hi)
{
  int h, l, p, t;

  if (lo < hi) {
    l = lo;
    h = hi;
    p = a[hi];

    do {
      while ((l < h) && (a[l] <= p))
          l = l+1;
      while ((h > l) && (a[h] >= p))
          h = h-1;
      if (l < h) {
          t = a[l];
          a[l] = a[h];
          a[h] = t;
      }
    } while (l < h);

    a[hi] = a[l];
    a[l] = p;

    f( a, lo, l-1 );
    f( a, l+1, hi );
  }
\end{lstlisting}

\end{frame}

\begin{frame}[fragile]
\frametitle{The same in Haskell}

\begin{lstlisting}
  qsort []     = []
  qsort (p:xs) = (qsort lesser) ++ [p] ++ (qsort greater)
    where
        lesser  = filter (< p) xs
        greater = filter (>= p) xs

\end{lstlisting}

\end{frame}

\begin{frame}
\frametitle{Motivation}

 \begin{itemize}
  \item 10 time less lines than in C,
  \item Great expressivity,
  \item Great genericity,
  \item More readable, more maintainable,
  \item Very few bugs: if it compiles, 90\% chance it works on the first try
 \end{itemize}

\end{frame}


\section{Features}
\begin{frame}
\frametitle{Features}

 \begin{itemize}
  \item Statically typed
  \item Functional
  \item Pure
  \item Lazy
 \end{itemize}

\end{frame}

\begin{frame}[fragile]
  \frametitle{Coming from Other Languages}
  \begin{description}

  \item[C, C++, Java] The notion of classes is almost completely different, return doesn't mean what you think it does, you don't get loops, and code doesn't run in the order it's typed on the screen.

  \item[Python, Perl] You might be in the same boat but you have functions like reduce, map, etc. that use functions as first-class objects, and abstract away loops. But if you write Perl like a C coder rather than a Lisp coder be prepared.

    \end{description}
\end{frame}

\begin{frame}[fragile]
  \frametitle{Coming from Other Languages}
  \begin{description}

  \item[Scheme, Lisp] You'll probably have a good start on the basic functional design idioms. What will drive you insane will be the type system.

  \item[ML, OCaml] You've still got a few things to learn, but these languages aren't that different. The main distinction (which is not small) is Haskell's purity, which will also be freaking out everyone in the above groups.

  \end{description}
\end{frame}

\subsection{Static Typing}

\begin{frame}[fragile]
  \frametitle{Static Typing}
  \begin{itemize}
  \item Type inference
  \item Static typing
    \begin{itemize}
    \item Declare your intentions to the compiler
    \item Filter bugs at compile time
    \end{itemize}
  \end{itemize}
\end{frame}

\begin{frame}[fragile]
  \frametitle{Defining functions}
  Everything has a type so a function must have a type.
  \uncover<1->{
    $$
    \begin{aligned}
      f\, x             &= x + 1 \\
    \end{aligned}
    $$
  }
  What is the type?
  \uncover<2->{
  $$
  \begin{aligned}
    \textrm{fact}\, 0 &= 1 \\
    \textrm{fact}\, n &= n \times \textrm{fact} (n - 1) \\
  \end{aligned}
  $$
  What is the type?
  }
\end{frame}

\begin{frame}[fragile]
  \frametitle{Partial Application}
  \uncover<1->{
    What is the type of $+$?
  }
  \uncover<2->{
  $$
  \begin{aligned}
    (+) &:: \textrm{Num} \, a \implies a \longrightarrow a \longrightarrow a \\
    (+) &:: \textrm{Num} \, a \implies a \longrightarrow (a \longrightarrow a) \\
  \end{aligned}
  $$
  }
  \uncover<3->{
  $$
    \begin{aligned}
      (+) &:: \textrm{Num} \, a \implies a \longrightarrow (a \longrightarrow a) \\
      (+)\, 1 &:: \textrm{Num} \, a \implies a \longrightarrow a \\
      (+ 1) &:: \textrm{Num} \, a \implies a \longrightarrow a \\
  \end{aligned}
  $$
  }
\end{frame}

\begin{frame}[fragile]
  \frametitle{Lists}
  \uncover<1->{
    Lists hold a somewhat special status (as do tuples).
  }
  \uncover<2->{
  $$
  \begin{aligned}
    \big[\big] &:: [a] \\
    (:)        &::  a \longrightarrow [a] \longrightarrow [a] \\
    (++)       &::  [a] \longrightarrow [a] \longrightarrow [a] \\
  \end{aligned}
  $$
  }
\end{frame}

\subsection{Functional}
\begin{frame}[fragile]
\frametitle{Functions are First Class}

\uncover<1->{
Suppose we want to apply a function to a list.
}

\uncover<2->{
$$
\textrm{map} :: (a \longrightarrow b) \longrightarrow [a] \longrightarrow [b]
$$
}

\uncover<3->{
$$
\textrm{map}\, (+1) :: \textrm{Num}\, a \implies [a] \longrightarrow [a]
$$
}

\end{frame}

\begin{frame}[fragile]
\frametitle{Higher Order Functions}

\uncover<1->{
Suppose we want to reduce a list of numbers to a single number.
}

\uncover<2->{
$$
[1,2,3,4,5] \equiv 1:2:3:4:5:[]
$$
}

\uncover<3->{
$$
1:2:3:4:5:[] \leadsto 1 \oplus 2 \oplus 3 \oplus 4 \oplus 5 \oplus z
$$
}

\uncover<4->{
$$
1:2:3:4:5:[] \leadsto 1 \oplus (2 \oplus (3 \oplus (4 \oplus (5 \oplus e))))
$$
}

\end{frame}

\framedgraphic{Higher Order Functions}{figs/Right-fold-transformation.png}

\begin{frame}[fragile]
\frametitle{Higher Order Functions}

\uncover<1->{
Suppose we want to sum a list of numbers.
}

\uncover<2->{
$$
  \textrm{foldr} :: (a \longrightarrow b \longrightarrow b)
         \longrightarrow b \longrightarrow [a] \longrightarrow b
$$
}

\uncover<3->{
$$
\textrm{foldr}\, (+) :: \textrm{Num}\, b \implies b \longrightarrow [b] \longrightarrow b
$$
}

\uncover<4->{
$$
\textrm{foldr}\, (+)\, 0:: \textrm{Num}\, b \implies [b] \longrightarrow b
$$
}

\end{frame}

\begin{frame}[fragile]
\frametitle{Higher Order Functions}

\uncover<1->{
Suppose we want implement factorial without recursion.
}

\uncover<2->{
$$
  \textrm{foldr} :: (a \longrightarrow b \longrightarrow b)
         \longrightarrow b \longrightarrow [a] \longrightarrow b
$$
}

\uncover<3->{
$$
\textrm{foldr}\, (*) :: \textrm{Num}\, b \implies b \longrightarrow [b] \longrightarrow b
$$
}

\uncover<4->{
$$
\textrm{foldr}\, (*)\, 1:: \textrm{Num}\, b \implies [b] \longrightarrow b
$$
}

\end{frame}

\begin{frame}[fragile]
\frametitle{Exercises 1}

Implement the following using \texttt{foldr} arranged in increasing difficulty.

 \vspace{0.5cm}
 \begin{block}{Exercises}

  \begin{lstlisting}
   myLength :: [a] -> Int

   dupElems :: [a] -> [a]
   dupElems ['a', 'b', 'c'] = ['a', 'a', 'b', 'b', 'c', 'c']

   myMap :: (a -> b) -> [a] -> [b]
  \end{lstlisting}

 \end{block}

\end{frame}

\subsection{Lazy}
\begin{frame}[fragile]
\frametitle{Lazy}

\begin{itemize}
\item A value is not calculated if it is not used
\item Infinite data structures
\item Better design for programs
\item Memoization
\end{itemize}
\vspace{0.5cm}
\begin{block}{Example}
  \begin{lstlisting}[basicstyle=\small]
    ones = 1 : ones
    evens = 0 : map (+2) evens
    powersOf2 = 1 : map (*2) powersOf2
  \end{lstlisting}
\end{block}

\end{frame}

\begin{frame}[fragile]
\frametitle{Not Wrong but Not Good}

\vspace{0.1cm}
\begin{block}{What is Bad?}
  \begin{lstlisting}[basicstyle=\small]
    fibs 0 = 0
    fibs 1 = 1
    fibs n = fibs (n - 1) + fibs (n - 2)
  \end{lstlisting}
\end{block}

\end{frame}

\begin{frame}[fragile]
\frametitle{Better}

$$
\begin{aligned}
  \textrm{zipWith} &:: (a \longrightarrow b \longrightarrow c) \longrightarrow [a] \longrightarrow [b] \longrightarrow [c] \\
  \textrm{tail} &:: [a] \longrightarrow [a]
\end{aligned}
$$
\vspace{0.1cm}
\begin{block}{Example}
  $$
  \begin{array}{rrrrrrrr}
    0 & 1 & 1 & 2 & 3 & 5 & 8 & \ldots \\
    + & + & + & + & + & + & + & \ldots \\
    1 & 1 & 2 & 3 & 5 & 8 & 13 & \ldots \\
    \downarrow & \downarrow & \downarrow & \downarrow & \downarrow & \downarrow & \downarrow & \downarrow \\
    1 & 2 & 3 & 5 & 8 & 13 & 21 & \ldots \\
  \end{array}
  $$
  \begin{lstlisting}[basicstyle=\small]
    fibs = 0 : 1 : (zipWith (+) fibs (tail fibs))
  \end{lstlisting}
\end{block}

\end{frame}


\begin{frame}[fragile]
\frametitle{Unfolding}

$$
\begin{aligned}
  \textrm{data Maybe}\, a &= \textrm{Nothing}\, |\, \textrm{Just}\, a \\
  \textrm{lookup} &:: \textrm{Eq}\, a \implies a \longrightarrow [(a, b)] \longrightarrow \textrm{Maybe}\, b \\
  \textrm{unfoldr} &:: (b \longrightarrow \textrm{Maybe}\, (a,b)) \longrightarrow b \longrightarrow [a] \\
  \textrm{take} &:: [a] \longrightarrow [a] \\
  \textrm{drop} &:: [a] \longrightarrow [a] \\
\end{aligned}
$$
\vspace{0.1cm}
\begin{block}{Example}
  \begin{lstlisting}[basicstyle=\small]
    toBinary :: Integer -> [Integer]
    toBinary = unfoldr g
      where
        g :: Integer -> Maybe (Integer, Integer)
        g 0 = Nothing
        g n = Just (n `mod` 2, n `div` 2)
  \end{lstlisting}
\end{block}

\end{frame}

\begin{frame}[fragile]
\frametitle{Exercises 2}

Implement the following using \texttt{unfoldr} arranged in increasing difficulty.

 \vspace{0.5cm}
 \begin{block}{Exercises}

  \begin{lstlisting}
   chunk :: Int -> [a] -> [[a]]
   chunk 4 [1..10] = [[1,2,3,4],[5,6,7,8],[9,10]]

   merge :: ([a], [a], [a]) -> [a]
   merge ([1..3], [11..13], [21..23]) = [1,11,21,2,12,22,3]
  \end{lstlisting}

 \end{block}

\end{frame}

\subsection{More on Types}
\begin{frame}[fragile]
\frametitle{Functors}

\begin{itemize}
\item Generalize \texttt{map}
\item Types have kinds
\item E.g. \texttt{[] :: * -> *}
\item \texttt{[]} is a type constructor i.e. it takes a type and returns a type
\end{itemize}

\begin{block}{Recall}
  \begin{lstlisting}[language=Haskell]
      map :: (a -> b) -> [a] -> [b]
  \end{lstlisting}
 \end{block}

\begin{block}{Generalize}
  \begin{lstlisting}[language=Haskell]
    class MyFunctor (f :: * -> *) where
      myFmap :: (a -> b) -> f a -> f b
  \end{lstlisting}
 \end{block}

\end{frame}

\begin{frame}[fragile]
  \frametitle{Functors}

\begin{block}{Maybe}
  \begin{lstlisting}[language=Haskell]
    instance MyFunctor Maybe where
    myFmap f Nothing  = Nothing
    myFmap f (Just x) = Just (f x)
  \end{lstlisting}
\end{block}

\begin{block}{List}
  \begin{lstlisting}[language=Haskell]
    instance MyFunctor [] where
    myFmap = map
  \end{lstlisting}
\end{block}

\end{frame}

\begin{frame}[fragile]
  \frametitle{Monads}

  \begin{block}{Function Composition}
    \begin{lstlisting}[language=Haskell]
      (.) :: (b -> c) -> (a -> b) -> a -> c
      f = (+1). (+1)
      g = take 10 . drop 10
    \end{lstlisting}
  \end{block}

\begin{block}{Definition}
  \begin{lstlisting}[language=Haskell]
    class MyFunctor m => MyMonad (m :: * -> *) where
    unit :: a -> m a
    (>=>) :: (a -> m b) -> (b -> m c) -> (a -> m c)
  \end{lstlisting}
\end{block}

\end{frame}


\begin{frame}[fragile]
\frametitle{Pure}
Do you really need for loops?

 \begin{block}{Java}
  \begin{lstlisting}[language=Java]
    for(int i = 0; i < 10; i++) {
       list[i] = list[i] + 1
    }

    int total = 0;
    for(int i = 0; i < 10; i++) {
       total = total + list[i]
    }

  \end{lstlisting}
 \end{block}

 \begin{block}{Haskell}
  \begin{lstlisting}[language=Haskell]
    map (+1) list
    foldr 0 (+) list
    sum list
  \end{lstlisting}
 \end{block}

\end{frame}

\begin{frame}
\frametitle{Pure}

What do I win with purity?
 \vspace{0.5cm}
 \begin{itemize}
  \item Much less bugs
  \item Easier to reason with
  \item Easier to refactor
  \item Easier to parallelize
  \item Enables equational reasoning
  \item Enables laziness
 \end{itemize}

\end{frame}



\section{IO}
\begin{frame}
\frametitle{IO}

 How do we ever perform IO if every function is pure?
  \vspace{1cm}
  \begin{columns}[]
   \begin{column}[]{5cm}
    \includegraphics[width=1\linewidth]{figs/pure}
   \end{column}
   \begin{column}[]{5cm}
    \includegraphics[width=1.1\linewidth]{figs/readWrite}
   \end{column}
  \end{columns}

\end{frame}

\begin{frame}
\frametitle{IO}
 Passing the world around
 \includegraphics[width=1\linewidth]{figs/world}
\end{frame}

\begin{frame}
\frametitle{IO}
 IO operations can now be chained!
 \includegraphics[width=1\linewidth]{figs/worldChain}

\end{frame}

\begin{frame}[fragile]
\frametitle{IO}
 The run-time is reading lazily the IO instructions from the output of the chain, and perform them.
 \includegraphics[width=0.8\linewidth]{figs/runTime}
 \begin{block}{Example}
  \begin{lstlisting}[basicstyle=\small]
   main = do
     putStrLn Foo
     putStrLn Bar
  \end{lstlisting}
 \end{block}

\end{frame}

\begin{frame}[fragile]
\frametitle{IO}
 readString and writeString can also interact because of lazyness.
 \vspace{0.5cm}
 \includegraphics[width=1\linewidth]{figs/pureReadWrite}
 \vspace{0.5cm}
 \begin{block}{Example}
  \begin{lstlisting}[basicstyle=\small]
   main = do
     a <- readStrLn
     putStrLn a
  \end{lstlisting}
 \end{block}
\end{frame}

\begin{frame}
\frametitle{Humor}
 \includegraphics[width=1\linewidth]{figs/langs}
\end{frame}

%\begin{frame}[fragile]
%\frametitle{IO}
%
% \begin{diagram}[width=300,height=200]
%  {-# LANGUAGE FlexibleContexts #-}
%  import Data.List
%  import Diagrams.TwoD.Types
%
%  inout = arrow 1
%  box = square 1 # fc green # named "box"
%
%  shaft  = cubicSpline False ( map p2 [(0, 0), (1, 0), (1, 0.2), (2, 0.2)])
%
%  inIO = translate (2 ^& 1) $ rotate (90 @@ deg) $ arrow' (with & arrowShaft .~ shaft) 1
%
%  dia = inIO <> (inout ||| box ||| inout) # pad 1.3
% \end{diagram}
%
%\end{frame}


%  Consolidate application/services and turn unused servers off:
%  \includegraphics[width=0.6\linewidth]{figs/VMmigration1}	\\
%  \vspace{1\baselineskip}
%  Relocate application/services to efficient servers:
%  \includegraphics[width=0.6\linewidth]{figs/VMmigration2}

\end{document}
